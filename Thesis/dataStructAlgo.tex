\chapter{Data Structure and Algorithms}
\section{A Relational Database}
For anyone handling a large, diverse dataset, databases like Oracle, 
PostgreSQL, and Microsoft Access are commonly used to store the data. 
The central idea of object oriented programming is to consider 
everything as an object, and to provide every object with enough 
member functions that it can interact with the outside world in a code 
in an easy to understand manner. So, why don't businesses use object 
oriented structures to store their data? What are their databases, and 
why are they superior?

While working on this model I experimented with several data 
structures, but none were quite suited to the task. The model 
described in this paper is a vertex dynamics model, and the vertices 
move based on inter- and intra- cellular forces. At once we want the 
object of focus to be a vertex, and every vertex to have some data 
member upon which it depends. The logical next step is to make a class 
called \textbf{vertex}, and make it have cells as member variables, 
since the forces acting on a vertex come from a cell. This model is 
impossible, however, because cells are nothing more than vertices 
with edges. A box which contains a box cannot be contained in that 
box. 

Finally, I decided to use a  \textbf{relational database}
structure. In this structure all data is stored across various 
tables, and the tables are connected based upon some key which relates 
the elements. Data is extracted from this type of structure by 
performing \textbf{joins} on the tables which contain the desired 
information.

Of course, C++ does not have a container class called table, but I was 
able to reproduce the table behaviour in my code. My code features 
three principal structures which can be thought of as tables. These 
structures are the simulationcells and coordinate vectors and the x 
and y vectors. The simulation cells vector contains all of the cells 
in the mesh, and each cell has a member that resembles a key in its 
AssociatedVertices member. This member has a list of vertex indices 
(which are called a primary key in the language of relational 
databases) which relates it to the coordinate vector. The coordinate 
vector contains all of the coordinates in the mesh, and their data can 
be extracted by their index. The coordinate class does not directly 
contain the position of a coordinate in it - this data is stored in 
the x and y vector (in the parallel version of the code), and is 
accessed via a pointer from the coordinate vector. By storing the x 
and y coordinates in a continuous block of memory instead of scattered 
through various instances of the coordinate class, parallelization of 
the updating of coordinate locations is faster. There will be more on 
this later.

% Here insert a graphic of the organization of the data structure.


\section{Calculating the Gradient}
\section{Moving the Vertices}
\section{The C++ Standard Library, and Use of the New Library.}
\section{Embarassing Parallelism and CUDA Thrust}
